\documentclass[10pt]{article}
\usepackage[utf8]{inputenc}
\usepackage[papersize={17in, 11in}]{geometry}
\usepackage[absolute]{textpos}
\TPGrid[0.5in, 0.25in]{23}{24}
\usepackage{nopageno}
\usepackage{palatino}
\usepackage{tabu}
\parindent=0pt
\parskip=12pt
\tabulinesep=1.0mm
\begin{document}

\begin{textblock}{23}(0, 1)
\center \huge PREFACE
\end{textblock}

\begin{textblock}{23}(0, 3)

\textbf{Akasha} is a music of invisibility, electricity and the open expanse of
the sky. The title is the Sanskrit word for the \ae ther, a concept once
understood as an unseen force present in all things in motion in the world.

\textbf{Scordatura.} The violins are tuned as usual. String IV of the viola is
tuned down a minor third to A$\natural$2; string IV of the cello is tuned down
a minor third to A$\natural$1.

\textbf{Accidentals.} Accidentals govern only one note. \textbf{Because of this
no natural signs appear in the score}: G$\sharp$4 G4 should be played
G$\sharp$4 G$\natural$4. This is especially important in the densely chromatic
sections that appear throughout the score.

\textbf{String contact points.} Five string contact points appear in the score:

\begin{tabu}{l l l}
\phantom{M} & XT & as close to the fingers as possible (without touching the fingers) \\
            & tasto & very noticeably tasto in color\\
            & pos. ord. & ordinary playing position \\
            & pont. & very noticeably ponticello in color \\
            & XP & as close to the bridge as possible (without touching the bridge) \\
\end{tabu}

\textbf{Bridge contact points.} The indication \textbf{OB} stands for
``directly on the bridge'' and means that the bow should be run diagonally on
the bridge to produce white noise with no pitch at all. Fractional bridge
contact points also appear. These are played with the bow extremely high on the
string such that the hair of the bow runs against both the wrapping of the
string and the wood of the bridge at the same time. Taken as a series these
bridge contact points do three things: they reduce the
fundamental of the string's fingered pitch; they increase the spectral content
of the upper partials; and they replace the overall sensation of pitch with
noise. Some examples:

\begin{tabu}{l l l}
\phantom{M} & XP & as close to the bridge as possible (without touching the bridge) \\
            & $\frac{1}{4}$OB & one quarter of the hair on bridge (and three quarters of the hair on string) \\
            & $\frac{1}{2}$OB & one half of the hair on bridge (and one half of the hair on string) \\
            & $\frac{3}{4}$OB & three quarters of the hair on bridge (and one quarter of the hair on string) \\
            & OB & bow directly on bridge with a diagonal bow (to produce white noise only) \\
\end{tabu}

\textbf{Bow speed colors.} The score contrasts widely different speeds of the bow:
 
\begin{tabu}{l l l}
\phantom{M} & XFB & extremely fast bow (extreme flautando with the bow only very lightly skimming the string) \\
            & FB & fast bow (very pronounced flautando just slightly less than above) \\
            & NBS & normal bow speed (neither flautando nor scratch) \\
            & $\frac{1}{4}$ scratch & timbre with one quarter part scratch (and three quarter parts pitch) \\
            & $\frac{1}{2}$ scratch & timbre with one half part scratch (and one half part pitch) \\
            & $\frac{3}{4}$ scratch & timbre with three quarter parts scratch (and one quarter part pitch) \\
            & scratch moltiss. & timbre with as much scratch (and as little pitch) as possible (though without encouraging subtones) \\
\end{tabu}

Do not substitute tasto for the FB and XFB degrees of bow speed flautando
requested in the score: bow speeds combine freely with the string and bridge
contact points given above. Indications for individuated clicks of the bow also
appear; these result from almost impossibly slow motions of the bow against the
string.

\textbf{All passages marked ``leggierissimo'' should be played off string.} The
effect is to be an incredibly fast, and nimble, flurry of notes. (All such
passages carry a quiet dynamic and are marked with staccati.)

\textbf{Glissandi.} Do not rearticulate note-heads in the middle of glissandi.

\end{textblock}

\begin{textblock}{23}(0, 23)

\textbf{Akasha} was written for the JACK Quartet who gave the world premiere
the piece on February 6\textsuperscript{th} 2016 in Paine Hall on the campus of
Harvard University.

\end{textblock}

\end{document}