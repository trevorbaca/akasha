\documentclass[10pt]{article}
\usepackage[utf8]{inputenc}
\usepackage[papersize={17in, 11in}]{geometry}
\usepackage[absolute]{textpos}
\TPGrid[0.5in, 0.25in]{23}{24}
\usepackage{palatino}
\parindent=0pt
\parskip=12pt
\usepackage{nopageno}
\begin{document}

\begin{textblock}{23}(0, 1)
\center \huge PREFACE
\end{textblock}

\begin{textblock}{23}(0, 3)

\textbf{Accidentals.} Accidentals govern only one note. Because of this no
natural signs appear in the score: G$\sharp$4 G4 should be played G$\sharp$4
G$\natural$4.

\textbf{Scordatura.} The violins are tuned as usual. String IV of the viola is
tuned to A$\natural$2 (a minor third lower than the usual tuning of
C$\natural$3). String IV of the cello is tuned to A$\natural$1 (a minor third
lower than the usual tuning of C$\natural$2).

\end{textblock}

\begin{textblock}{23}(0, 23)

\textbf{Akasha} was written for the JACK Quartet. The piece is to be premiered
by the JACK Quartet on 6 February 2016 in Paine Hall on the campus of Harvard
University.

\end{textblock}

\end{document}